\documentclass[a4paper]{article}
\usepackage[UTF8]{ctex} % 加载ctex包

\usepackage[utf8]{inputenc} % allow utf-8 input
\usepackage[T1]{fontenc}    % use 8-bit T1 fonts
\usepackage{hyperref}       % hyperlinks
\usepackage{url}            % simple URL typesetting
\usepackage{booktabs}       % professional-quality tables


\usepackage{amsfonts}       % blackboard math symbols


\usepackage{nicefrac}       % compact symbols for 1/2, etc.
\usepackage{microtype}      % microtypography
\usepackage{lipsum}
\usepackage{geometry}
\geometry{a4paper,left=3.18cm,right=3.18cm,top=2.54cm,bottom=2.54cm} % Word标准页面
\usepackage{graphicx}
\newcommand{\song}{\CJKfamily{song}}    % 宋体
\fontsize{12pt}{18pt}\selectfont    % 小四,1.5倍行距
\hypersetup{
    colorlinks=true,
    linkcolor=black
}   % 去除目录文字红色边框

\title{} % 题目

\begin{document}
    % 设置封面
    \begin{center}  % 居中
        \thispagestyle{empty} % 当前页不显示页码
       
        \includegraphics[width=3in]{封面.png} % 封面图片
       
        \vspace{2cm} % 设置距离

        {\heiti  \fontsize{48pt}{1.5} \textbf{加粗的封面标题(黑体)}}

        \vspace{4cm} % 设置距离

        {\songti \huge \textbf{论文标题(宋体)}}

        \vspace{6cm} % 设置距离

        {\heiti  \fontsize{15pt}{1} \textbf{学科名称:\underline{\qquad XXXX \qquad}}}

        {\heiti  \fontsize{15pt}{1} \textbf{任课教师:\underline{\qquad  XXX(两个字可以用\qquad隔开)  \qquad}}}



        {\heiti  \fontsize{15pt}{1} \textbf{学\quad \ \ 号:\underline{\quad  XXXXXXXX \ \quad}}}



        {\heiti  \fontsize{15pt}{1} \textbf{姓\quad \ \ 名:\underline{\qquad XXX \qquad}}}

    \end{center}
\maketitle
\thispagestyle{empty} % 当前页不显示页码

%%%%%%%%%%%%%摘要与关键词%%%%%%%%%%%%%
\begin{abstract}
XXXX
\end{abstract} % 或者:\par\textbf{摘要:}XXXXXX
\par\textbf{关键词:}XXXX

\clearpage  % 新建页

\thispagestyle{empty} % 当前页不显示页码

%%%%%%%%%%%%%%%%%%%%%目录%%%%%%%%%%%%%%%%%%
\tableofcontents    % 目录

\clearpage  % 新建页

\setcounter{page}{1} % 从下面开始编页码

%%%%%%%%%%%%%%%%%%%%%正文%%%%%%%%%%%%%%%%%%
\section{XXXXX} % 第一节
\subsection{XXXXX}
XXX
\subsection{XXXXX}
XXX
\subsection{XXXXX}
XXX
% ……
\section{XXXXX} % 第二节
\subsection{XXXXX}
XXX
\subsection{XXXXX}
XXX
\subsection{XXXXX}
XXX
% ……
\section{XXXXX} % 第三节
\subsection{XXXXX}
XXX
\subsection{XXXXX}
XXX
\subsection{XXXXX}
XXX
% ……

%%%%%%%%%%%%%%%%%%%%%%%%%%%%%%%%%%%%%% 公式%%%%%%%%%%%%%%%%%%%%%%%%%%
$$
lin_x
\eqno (1)
$$

%%%%%%%%%%%%%%%%%%插入图片%%%%%%%%%%%%%%

\begin{figure}[h] %H为当前位置,!htb为忽略美学标准,htbp为浮动图形
    \centering %图片居中
    \includegraphics[width=1\textwidth]{XXX} %插入图片,[]中设置图片大小,{}中是图片文件名
    \caption{XXX} %最终文档中显示的图片标题
    \label{Fig.main3} %用于文内引用的标签
\end{figure}

%%%%%%%%%%%%%%%%%%%%%%%%%%%%%%%%%%参考文献%%%%%%%%%%%%%%%%%%%%%%%%%
% 参考文献,bib文件引入,bib文件放在同一文件夹,可以用“百度学术”或者“谷歌学术”生成
\clearpage
\nocite{*} %显示所有文献(你懂得)
\bibliographystyle{plain}
\bibliography{ref.bib}

\end{document}

% \par命令另起一行
% $$行内公式,$$$$行间公式